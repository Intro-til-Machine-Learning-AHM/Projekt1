\section{Project 1}

\subsection{1. A description of your data set.}

The dataset is composed by an examination of Pima Indians in Phoenix, Arizona.
They were chosen as subject due to their high risk of getting diabetes.
The dataset was used as a way to predict if Indian women could develop diabetes in the following 5 years.
The examination had picked out a number of attributes which they found relevant to the examination.
The data was picked out by: “Jack W. Smith, et al., 1988, Using the ADAP Learning Algorithm to Forecast the Onset of Diabetes Mellitus”
and is taken from a bigger examination made by “National Institute of Diabetes and Digestive and Kidney Diseases.”
The dataset contains 8 attributes, plus a class attribute which identifies whether or not the person was tested positive for diabetes, and the test was taken by 768 test subjects.

The dataset is composed by the following attributes:
\begin{enumerate}
\item Number of times pregnant
\item Plasma glucose concentration a 2 hours in an oral glucose tolerance test, GTT
\item Diastolic blood pressure mm Hg
\item Triceps skin fold thickness mm
\item 2-Hour serum insulin $\mu$ U / ml
\item Body mass index Weight in kg / (Height in $m^2$)
\item Diabetes pedigree function
\item Age Years

\item \emph{class attribute}:
\item Class variable 0 or 1
\end{enumerate}


The dataset was found on the website: (http://archive.ics.uci.edu/ml/datasets/Pima%20Indians%20Diabetes)
In the folder "Data Folder".

We want to determine whether or not the Indian women have diabetes or not, by using binary classification.
Because the outcome will be if they either have diabetes or if they do not.


\subsection{2. A detailed explanation of the attributes of the data.}

\subsubsection{Underunderoverskrift 1}


\subsection{Attributes of the data}
This dataset contains 8 attributes which work as input. These attributes are defined in
the above section and is discusses more in depth in the following.

The number of times pregnant is a discrete ratio attribute and so is
the age attribute. Both attributes are measured in a finite set of values.
Plasma glucose concentration is a continuous ratio attribute
and so is diastolic blood pressure, skin fold thickness, BMI and 2-hour serum
insulin. This means that only one of the eight input attributes is non-ratio.
The diabetes pedigree function is a continous ordinal attribute, because these
numbers tell whether one is more or less genetically expected to be diagnosed
with diabetes.
\bigskip

\begin{description}
\item [Number of times pregnant] Discrete ratio
\item [Plamsa glucose concentration] Continous ratio
\item [Diastolic blood pressure] Continous ratio
\item [Triceps skin fold] Continous ratio
\item [2-Hour serum insulin] Continous ratio
\item [Body mass index (BMI)] Continous ratio
\item [Diabetes pedigree function] Continous ordinal
\item [Age] Discrete ratio
\end{description}


Some of the recorded data make no biological sense and observations with one or
more nonsense values in its attributes have been discarded from the dataset.
Common sense has been applied to filter out obviously foul observations, such
as entries that dictate concentrations of blood plasma glucose to be zero, blood
pressure of zero, skin fold thickness as zero and BMI of zero. This means a
considerable amount of the original observations have been discarded, as the
data set went from 768 observations to 532.
\bigskip

Some summary statistics have been summarized in the tables below. This should
give some insight to the values usually seen in the attributes. Mean, variance
and correalation have been presented for the attributes. Here, the huge differences
in the variance of the attributes should be noted. Some are very high, some are
almost non-existent.

\begin{table}[]
\centering
\caption{Mean and median of attributes}
\label{my-label}
\begin{tabular}{lllllllll}
\cline{1-1}
\multicolumn{1}{|l|}{Attribute} & pregs & glucose & bp   & thickness & insulin & bmi  & dia\_pedig & age  \\ \cline{1-1}
\multicolumn{1}{|l|}{Mean}      & 3.3   & 122.6   & 70.7 & 29.1      & 156.1   & 33.0 & 0.5        & 30.9 \\ \cline{1-1}
\multicolumn{1}{|l|}{Variance}  & 10.3     & 950.0     & 155.8   & 110.3        & 14087.3    & 49.3 & 0.1       & 103.8 \\ \cline{1-1}
                                                       &       &         &      &           &         &      &            &      \\ \hline
\end{tabular}
\end{table}

\begin{table}
\caption{Correalation matrix}
\begin{tabular}{lrrrrrrrrr}
\toprule
{} &  pregs &  glucose &    bp &  thickness &  insulin &   bmi &  dia\_pedig &   age &  class \\
\midrule
pregs     &   1.00 &     0.20 &  0.21 &       0.09 &     0.08 & -0.03 &       0.01 &  0.68 &   0.26 \\
glucose   &   0.20 &     1.00 &  0.21 &       0.20 &     0.58 &  0.21 &       0.14 &  0.34 &   0.52 \\
bp        &   0.21 &     0.21 &  1.00 &       0.23 &     0.10 &  0.30 &      -0.02 &  0.30 &   0.19 \\
thickness &   0.09 &     0.20 &  0.23 &       1.00 &     0.18 &  0.66 &       0.16 &  0.17 &   0.26 \\
insulin   &   0.08 &     0.58 &  0.10 &       0.18 &     1.00 &  0.23 &       0.14 &  0.22 &   0.30 \\
bmi       &  -0.03 &     0.21 &  0.30 &       0.66 &     0.23 &  1.00 &       0.16 &  0.07 &   0.27 \\
dia\_pedig &   0.01 &     0.14 & -0.02 &       0.16 &     0.14 &  0.16 &       1.00 &  0.09 &   0.21 \\
age       &   0.68 &     0.34 &  0.30 &       0.17 &     0.22 &  0.07 &       0.09 &  1.00 &   0.35 \\
class     &   0.26 &     0.52 &  0.19 &       0.26 &     0.30 &  0.27 &       0.21 &  0.35 &   1.00 \\
\bottomrule
\end{tabular}
\end{table}

\subsection{Data Visualization}

The PCA revealed that a lot of the variance in the data is described by the
first principal component, the insulin. This principal component accounts for
93% of the variance, and together with the second principal component, 97% of
the variance in the data is explained. This means a lot of the variance can
be described in a 2D-plot. A cumulative sum can be seen on figure (??) as a
function of principal components, from principal component describing the most
variance to the principal component describing the least variance.

When the data is not standardized, one of the principal components and the one
with the most responsibility for the variance in the dataset is insulin. Therefore,
this is more or less the first principal direction in which we plot our data. The second
principal direction is much dictated by the attribute glucose, which is the
attribute with the second highest variance. In this way, the projected plot in
2D can more or less be though of as projection of the data down to the plane
in the insulin and glucose directions.
