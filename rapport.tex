\section{Overskrift 1}

Her er noget tekst under overskrift 1.

\subsection{Underoverskrift 1}

\subsubsection{Underunderoverskrift 1}

Noget tekst

Noget mere tekst


\subsection{Attributes of the data}
This dataset contains 8 attributes which work as input. These attributes are defined as follows
[Smith et al., 1988]
\begin{enumerate}
\item Number of times pregnant
\item Plasma glucose concentration
\item Diastolic blood pressure
\item Triceps skin fold thickness
\item 2-Hour serum insulin
\item Body Mass Index (BMI)
\item Diabetes pedigree function
\item Age
\end{enumerate}

The number of times pregnant is a discrete ratio attribute and so is
the age attribute. Both attributes are measured in a finite set of values.
Plasma glucose concentration is a continuous ratio attribute
and so is diastolic blood pressure, skin fold thickness, BMI and 2-hour serum
insulin. This means that only one of the eight input attributes is non-ratio.
The diabetes pedigree function is a continous ordinal attribute, because these
numbers tell whether one is more or less genetically expected to be diagnosed
with diabetes.
\bigskip

Some of the recorded data make no biological sense and observations with one or
more nonsense values in its attributes have been discarded from the dataset.
Common sense has been applied to filter out obviously foul observations, such
as entries that dictate concentrations of blood plasma glucose to be zero, blood
pressure of zero, skin fold thickness as zero and BMI of zero. This means a
considerable amount of the original observations have been discarded, as the
data set went from 768 observations to 532.
\bigskip
